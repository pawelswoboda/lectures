\subsection{Exact Inference with Variable Elimination}

\begin{frame}{Variable Elimination: Intuition}
    \begin{example}[Chain Graph BN]
    \begin{tikzpicture}[baseline={(A.base)}]
    \node[rand_var] (A) at (0,0) {$A$};
    \node[rand_var] (B) at (1,0) {$B$};
    \node[rand_var] (C) at (2,0) {$C$};
    \node[rand_var] (D) at (3,0) {$D$};
    \draw[->] (A) -- (B);
    \draw[->] (B) -- (C);
    \draw[->] (C) -- (D);
    \end{tikzpicture}

    \begin{itemize}
        \pause \item Factorization $\Pb[A,B,C,D] = \Pb[A] \Pb[B | A] \Pb[C | B] \Pb[D | C]$.
    \pause \item Goal: Compute $\Pb[B] \pause = \sum_a \Pb[a] \Pb[B | a]$. \pause All needed factors present in BN.
    \pause \item Goal: Compute $\Pb[C] \pause = \sum_b \Pb[b] \Pb[C | b]$. 
    \pause \item Goal: Compute $\Pb[D] \pause = \sum_c \Pb[c] \Pb[D | c]$. 
    \begin{itemize}
    \pause \item Complexity: $\mathcal{O}(k^2)$, where $k$ is number of states.
    \end{itemize}
    \pause \item General length $n$ chain complexity is \pause $\mathcal{O}(n k^2)$.
    \end{itemize}
    \end{example}
\pause
    \begin{example}[Full Graph BN]
    \begin{tikzpicture}[baseline={(A.base)}]
    \node[rand_var] (A) at (0,0) {$A$};
    \node[rand_var] (B) at (1,0) {$B$};
    \node[rand_var] (C) at (2,0) {$C$};
    \node[rand_var] (D) at (3,0) {$D$};
    \draw[->] (A) -- (B);
    \draw[->] (B) -- (C);
    \draw[->] (C) -- (D);
    \draw[->, bend left=30] (A) to (C);
    \draw[->, bend left=40] (A) to (D);
    \draw[->, bend left=30] (B) to (D);
    \end{tikzpicture}
    \begin{itemize}
        \pause \item Factorization $\Pb[A,B,C,D] = \Pb[A] \Pb[B | A] \Pb[C | A, B] \Pb[D | A,B,C]$.
        \pause \item Compute $\Pb[B]$ as above, $\Pb[C] = \sum_{a,b} \Pb[C | a,b] \Pb[b | a] \Pb[a]$.
        \pause \item $\Pb[D] = \sum_{a,b,c} \Pb[D | a,b,c] \Pb[c | a, b] \Pb[b | a] \Pb[a]$.
        \pause \item We cannot reuse computations as for chain graph.
    \pause \item Complexity using chain rule is \pause $\mathcal{O}(k^n)$!
    \end{itemize}
\end{example}
\end{frame}

\begin{frame}{Variable Elimination}
    \begin{itemize}
        \item Computing $\Pb[X]$ from $\Pb[X,Y]$ requires summing out all $Y$.
    \pause \item To this end: Define a basic operation summation operation on factors
%    \begin{itemize}
%    \item Given a factor $\phi(X,Y)$ (either in Bayesian or Markov network), obtain a marginal factor by summing out all $Y$.
%    \end{itemize}
    \end{itemize}
    \pause
    \begin{definition}[Factor Marginalization]
        Let $X$ be a set of random variables, $Y \notin X$.
        Let $\phi(X,Y)$ be a factor.
        We define the factor marginalization of $Y$ in $\phi$ as 
        \begin{equation}
            \psi(X) = \sum_Y \phi(X,Y)\,.
        \end{equation}
        We call this operation summing out $Y$.
    \end{definition}
    \pause
\begin{itemize}
    \item Idea for summing out variables in probability distribution: Do it one by one.
\end{itemize}
    \pause
    \begin{algorithm}[H]
        \caption{Sum-Product Variable Elimination}
        \vspace{-0.1cm}
        \KwInput{
            Factors $\Phi$,
            variable order $\prec$,
            variables to be eliminated $Z = \{Z_1 \prec \ldots \prec Z_k\}$.
            }
            \For{$i=1,\ldots,k$}{
                \begin{tcolorbox}[boxsep=0pt,width=5cm]
                $\Phi' \leftarrow \{ \phi \in \Phi: Z_i \in \Scope[\phi]\}$\;
                $\Phi'' \leftarrow \Phi \backslash \Phi'$\;
                $\psi \leftarrow \prod_{\phi \in \Phi'} \phi$\;
                $\tau \leftarrow \sum_{Z_i} \psi$\;
                $\Phi \leftarrow \Phi'' \cup \{\tau\}$\;
                \end{tcolorbox}
            }
            \KwOutput{$\Phi$}
        \label{alg:sum-product-variable-elimination}
    \end{algorithm}
\end{frame}

\begin{frame}{Sum-Product Application to Chain Graph}
    \begin{example}[Sum-Product on Chain Graph]
        Recall that the chain graph factorizes as
    \begin{equation}
    \Pb[A,B,C,D] = \phi_A \phi_{AB} \phi_{BC} \phi_{CD}
    \end{equation}
    \pause 
    Then application of Algorithm~\ref{alg:sum-product-variable-elimination} with $Z = \{A \prec B \prec C\}$ gives
    \begin{equation}
        \begin{aligned}
        \pause \Pb[D] = & \sum_{C} \sum_{B} \sum_{A} \phi_A \phi_{AB} \phi_{BC} \phi_{CD} \\
        \pause = & \sum_C \sum_B \phi_{CD} \phi_{BC} \underbrace{\left( \sum_A \phi_A \phi_{AB}\right)}_{= \tau_1, \text{Iteration } 1} \\
        \pause = & \underbrace{\sum_C \phi_{CD} \underbrace{\left(\sum_B \phi_{BC} \left( \sum_A \phi_A \phi_{AB}\right)\right)}_{= \tau_2, \text{Iteration } 2}}_{= \tau_3, \text{Iteration } 3} \\
        \end{aligned}
    \end{equation}
    \end{example}

    \pause
    \begin{itemize}
    \item Efficiency of Sum-Product relies on the fact that
    \begin{equation}
        \sum_X (\phi_1 \phi_2) = \phi_1 \sum_X \phi_2 \quad\quad \text{whenever } X \notin \Scope[\phi_1]\,.
    \end{equation}
    \end{itemize}
\end{frame}

\begin{frame}{Sum-Product Variable Elimination Correctness}
\begin{theorem}[Correctness of Algorithm~\ref{alg:sum-product-variable-elimination}]
Let $X$ be a set of random variables, $\Phi$ a set of factors with $\Scope[\phi] \subseteq X$ for $\phi \in \Phi$.
Let $Y \subseteq X$ be a set of query variables. For any ordering $\prec$ of $X \backslash Y$ Algorithm~\ref{alg:sum-product-variable-elimination}$(\Phi, X \backslash Y, \prec)$ returns a factor $\phi^*(Y)$ such that
\begin{equation}
    \phi^*(Y) = \sum_{X \backslash Y} \prod_{\phi \in \Phi} \phi\,.
\end{equation}
\end{theorem}
\pause
\begin{proof}
Immediate from the observation that Algorithm~\ref{alg:sum-product-variable-elimination} is just a reordering of summations.
\end{proof}
\begin{description}
    \pause \item[Bayesian Network:] Algorithm~\ref{alg:sum-product-variable-elimination} directly gives marginal probabilities:
\begin{equation}
    VE(\Phi, \prec, Z) = \prod_{X_i \notin Z} \Pb[X_i | \Pa{G}{X_i} \backslash Z]\,.
\end{equation}
    \pause \item[Markov Network:] Algorithm~\ref{alg:sum-product-variable-elimination} gives an unnormalized factor over the query. 
    \begin{equation}
    VE(\Phi, \prec, Z) = \sum_{X \in Z} \prod_{i} \phi_i 
    \end{equation}
    \begin{itemize}
        \pause \item Compute partition function to obtain a probability distribution via
        $Z = VE(\Phi, \prec, \mathcal{X})$.
    \end{itemize}
    \pause \item[Evidence:] Given evidence $E = e$ replace factors by $\phi[E = e]$ and apply Algorithm~\ref{alg:sum-product-variable-elimination}.
\end{description}
\end{frame}

\begin{frame}{Variable Elimination: Graph Transformation}
\begin{itemize}
    \item What happens to the factorization during Algorithm~\ref{alg:sum-product-variable-elimination}?
    \pause
    \begin{itemize}
        \item \underline{Graph viewpoint:} When Bayesian, moralize the graph.
    \end{itemize}
    \pause
    \item When a variable is eliminated in Algorithm~\ref{alg:sum-product-variable-elimination}:
    \begin{itemize}
    \pause \item Create factor $\psi$ with $\Scope[\psi] = \{X\} \cup \{Y : \{X,Y\} \in E\}$.
        \begin{itemize}
    \pause \item \underline{Graph viewpoint:} Add all edges in $\Scope[\psi]$ (called fill edges).
        \end{itemize}
    \pause \item Sum out $X$ from $\psi$, obtain factor $\tau$ with $\Scope[\tau] = \Scope[\psi] \backslash \{X\}$.
        \begin{itemize}
    \pause \item \underline{Graph viewpoint:} Remove node $X$ and all adjacent edges.
        \end{itemize}
    \end{itemize}
\end{itemize}
\pause
\hrule
\begin{center}
\begin{tikzpicture}[scale=0.6]
    \node at (0.5,6) {Original graph};
    \node[rand_var] (C) at (-1, 5) {C};
    \node[rand_var] (D) at (-1, 4) {D};
    \node[rand_var] (G) at (0, 3) {G};
    \node[rand_var] (I) at (1, 4) {I};
    \node[rand_var] (L) at (0, 2) {L};
    \node[rand_var] (H) at (-1, 1) {H};
    \node[rand_var] (S) at (1, 1) {S};
    \node[rand_var] (J) at (2, 3) {J};
    \draw[->] (C) to (D);
    \draw[->] (D) to (G);
    \draw[->] (I) to (G);
    \draw[->] (I) to (J);
    \draw[->] (J) to (S);
    \draw[->] (G) to (L);
    \draw[->] (L) to (S);
    \draw[->] (S) to (H);
    \draw[bend right=15,->] (G) to (H);
\pause
    \begin{scope}[shift={(4,0)}]
    \node at (0.5,6) {Moral graph};
    \node[rand_var] (C) at (-1, 5) {C};
    \node[rand_var] (D) at (-1, 4) {D};
    \node[rand_var] (G) at (0, 3) {G};
    \node[rand_var] (I) at (1, 4) {I};
    \node[rand_var] (L) at (0, 2) {L};
    \node[rand_var] (H) at (-1, 1) {H};
    \node[rand_var] (S) at (1, 1) {S};
    \node[rand_var] (J) at (2, 3) {J};
    \draw (C) to (D);
    \draw (D) to (G);
    \draw[blue] (D) to (I);
    \draw (I) to (G);
    \draw (I) to (J);
    \draw (J) to (S);
    \draw (G) to (L);
    \draw[blue, bend left=15] (G) to (S);
    \draw (L) to (S);
    \draw[blue] (L) to (J);
    \draw (S) to (H);
    \draw[bend right=15] (G) to (H);
    \node at (0.5,-0.5) {\textcolor{blue}{$-$}: Moral edge};
    \end{scope}
\pause
    \begin{scope}[shift={(8,0)}]
    \node at (0.5,6) {Eliminate C};
    \node[gray!30, rand_var] (C) at (-1, 5) {C};
    \node[rand_var] (D) at (-1, 4) {D};
    \node[rand_var] (G) at (0, 3) {G};
    \node[rand_var] (I) at (1, 4) {I};
    \node[rand_var] (L) at (0, 2) {L};
    \node[rand_var] (H) at (-1, 1) {H};
    \node[rand_var] (S) at (1, 1) {S};
    \node[rand_var] (J) at (2, 3) {J};
    \draw[gray!30] (C) to (D);
    \draw (D) to (G);
    \draw[blue] (D) to (I);
    \draw (I) to (G);
    \draw (I) to (J);
    \draw (J) to (S);
    \draw (G) to (L);
    \draw[blue, bend left=15] (G) to (S);
    \draw (L) to (S);
    \draw[blue] (L) to (J);
    \draw (S) to (H);
    \draw[bend right=15] (G) to (H);
    \end{scope}
\pause
    \begin{scope}[shift={(12,0)}]
    \node at (0.5,6) {Eliminate D};
    \node[gray!30, rand_var] (C) at (-1, 5) {C};
    \node[gray!30, rand_var] (D) at (-1, 4) {D};
    \node[rand_var] (G) at (0, 3) {G};
    \node[rand_var] (I) at (1, 4) {I};
    \node[rand_var] (L) at (0, 2) {L};
    \node[rand_var] (H) at (-1, 1) {H};
    \node[rand_var] (S) at (1, 1) {S};
    \node[rand_var] (J) at (2, 3) {J};
    \draw[gray!30] (C) to (D);
    \draw[gray!30] (D) to (G);
    \draw[gray!30] (D) to (I);
    \draw (I) to (G);
    \draw[blue] (I) to (J);
    \draw (J) to (S);
    \draw (G) to (L);
    \draw[blue, bend left=15] (G) to (S);
    \draw (L) to (S);
    \draw[blue] (L) to (J);
    \draw (S) to (H);
    \draw[bend right=15] (G) to (H);
    \end{scope}
\pause
    \begin{scope}[shift={(16,0)}]
    \node at (0.5,6) {Eliminate I};
    \node[gray!30, rand_var] (C) at (-1, 5) {C};
    \node[gray!30, rand_var] (D) at (-1, 4) {D};
    \node[rand_var] (G) at (0, 3) {G};
    \node[gray!30, rand_var] (I) at (1, 4) {I};
    \node[rand_var] (L) at (0, 2) {L};
    \node[rand_var] (H) at (-1, 1) {H};
    \node[rand_var] (S) at (1, 1) {S};
    \node[rand_var] (J) at (2, 3) {J};
    \draw[gray!30] (C) to (D);
    \draw[gray!30] (D) to (G);
    \draw[gray!30] (D) to (I);
    \draw[gray!30] (I) to (G);
    \draw[gray!30] (I) to (J);
    \draw[red] (G) to (J);
    \draw (J) to (S);
    \draw (G) to (L);
    \draw[blue, bend left=15] (G) to (S);
    \draw (L) to (S);
    \draw[blue] (L) to (J);
    \draw (S) to (H);
    \draw[bend right=15] (G) to (H);
    \node at (0.5,-0.5) {\textcolor{red}{$-$}: Fill edge};
    \end{scope}
\end{tikzpicture}
\end{center}
\end{frame}

\begin{frame}{Variable Elimination: Induced Graph}
    \label{slide:induced-graph}
\begin{definition}[Induced Graph]
    Let $\Phi$ be a set of factors over $\mathcal{X}$ and $\prec$ be an elimination order.
    The induced graph $\mathcal{I}_{\Phi,\prec}$ is an undirected graph over $\mathcal{X}$ with $X_i - X_j$ if they appear in some intermediate factor $\psi$ generated by Algorithm~\ref{alg:sum-product-variable-elimination} using $\prec$ as elimination order.
\end{definition}
\pause
\begin{itemize}
    \item Elimination order: C $\prec$ D $\prec$ I $\prec$ H $\prec$ G $\prec$ S $\prec$ L.
\end{itemize}
\vspace{-0.4cm}
    \begin{minipage}[t]{0.29\textwidth}
\begin{example}
    \begin{center}
Ind.\ Graph $\mathcal{I}_{\Phi,\prec}$
\begin{tikzpicture}[scale=0.7,baseline={(C.base)}]
    \node[rand_var] (C) at (-1, 5) {C};
    \node[rand_var] (D) at (-1, 4) {D};
    \node[rand_var] (G) at (0, 3) {G};
    \node[rand_var] (I) at (1, 4) {I};
    \node[rand_var] (L) at (0, 2) {L};
    \node[rand_var] (H) at (-1, 1) {H};
    \node[rand_var] (S) at (1, 1) {S};
    \node[rand_var] (J) at (2, 3) {J};
    \draw (C) to (D);
    \draw(D) to (G);
    \draw[blue]  (D) to (I);
    \draw (I) to (G);
    \draw (I) to (J);
    \draw[red] (G) to (J);
    \draw (J) to (S);
    \draw (G) to (L);
    \draw[blue, bend left=15] (G) to (S);
    \draw (L) to (S);
    \draw[blue] (L) to (J);
    \draw (S) to (H);
    \draw[bend right=15] (G) to (H);
\end{tikzpicture}
\end{center}
\end{example}
\end{minipage}
\pause
\begin{minipage}[t]{0.29\textwidth}
\begin{example}
\begin{center}
Cliques of $\mathcal{I}_{\Phi,\prec}$
\begin{tikzpicture}[scale=0.7,baseline={(C.base)}]
    \node[rand_var] (C) at (-1, 5) {C};
    \node[rand_var] (D) at (-1, 4) {D};
    \node[rand_var] (G) at (0, 3) {G};
    \node[rand_var] (I) at (1, 4) {I};
    \node[rand_var] (L) at (0, 2) {L};
    \node[rand_var] (H) at (-1, 1) {H};
    \node[rand_var] (S) at (1, 1) {S};
    \node[rand_var] (J) at (2, 3) {J};
    \draw (C) to (D);
    \draw (D) to (G);
    \draw (D) to (I);
    \draw (I) to (G);
    \draw (I) to (J);
    \draw (G) to (J);
    \draw (J) to (S);
    \draw (G) to (L);
    \draw[bend left=15] (G) to (S);
    \draw (L) to (S);
    \draw (L) to (J);
    \draw (S) to (H);
    \draw[bend right=15] (G) to (H);
    \pause 
    \filldraw[fill=blue!70, nearly transparent, tension=0.7] plot[smooth cycle] coordinates {(C.north west) (D.south west) (D.south east) (C.north east)};
    \pause 
    \filldraw[fill=green!70, nearly transparent, tension=0.7] plot[smooth cycle] coordinates {(D.north west) (G.south) (I.north east)};
    \pause 
    \filldraw[fill=orange!70, nearly transparent, tension=0.7] plot[smooth cycle] coordinates {(I.north) (G.south west) (J.south east)};
    \pause 
    \filldraw[fill=red!50, nearly transparent, tension=0.7, even odd rule] plot[smooth cycle] coordinates {(H.south west) (S.east) (G.north)} (L) circle (0.55cm);
    \pause 
    \filldraw[fill=brown!50, nearly transparent, tension=0.7] plot[smooth cycle] coordinates {(G.north west) (L.south west) (S.south) (J.east) };
    \pause 
    \filldraw[fill=gray!70, nearly transparent, tension=0.7] plot[smooth cycle] coordinates {(S.south) (L.west) (J.north east)};
    \pause 
    \filldraw[fill=black!70, nearly transparent, tension=0.7] plot[smooth cycle] coordinates {(L.south) (J.south) (J.north) (L.north)};
\end{tikzpicture}
\end{center}
\end{example}
\end{minipage}
\pause
\begin{minipage}[t]{0.39\textwidth}
\begin{theorem}
Let $\mathcal{I}_{\Phi,\prec}$ be the induced graph, then:
\begin{itemize}
    \item $\Scope[\psi]$ of every factor $\psi$ generated during Algorithm~\ref{alg:sum-product-variable-elimination} is a clique in $\mathcal{I}_{\Phi,\prec}$.
    \item Every maximal clique in $\mathcal{I}_{\Phi,\prec}$ is the scope of an intermediate factor $\psi$ in the computation.
\end{itemize}
\end{theorem}
\pause
\vspace{-0.35cm}
\begin{proof}
First statement clear from construction.
\pause
For the second, consider the first time $X$ is visited in the maximal clique $C$ creating $\psi$.
\pause
Since $X$ is connected to any other node in the clique, $\Scope[\psi] = C$.
\end{proof}
\end{minipage}
%\textbf{(Maximal) Clique tree:}
%\begin{tikzpicture}[baseline={(CD.base)}]
%    \node[rand_var] (CD) at (0,0) {CD};
%    \node[rand_var] (GID) at (2,0) {GID};
%    \node[rand_var] (GIS) at (4,0) {GIS};
%    \node[rand_var] (GJSL) at (6,0) {GJSL};
%    \node[rand_var] (GHJ) at (8,0) {GHJ};
%    %
%    \draw (CD) to node[below] {D} (GID);
%    \draw (GID) to node[below] {G,I} (GIS);
%    \draw (GIS) to node[below] {G,S} (GJSL);
%    \draw (GJSL) to node[below] {G,J} (GHJ);
%\end{tikzpicture}
\end{frame}

\begin{frame}{Induced Width, Tree Width}
    \begin{definition}[Induced Width]
        The induced width of a graph is the maximal number of variables in a clique minus one.
        The induced width $w_{H,\prec}$ of an order $\prec$ and graph $H$ is the induced width of $\mathcal{I}_{H,\prec}$.
    \end{definition}
    \pause
    \begin{definition}[Tree Width]
       The tree width of a graph $H$ is the minimal induced width over all possible elimination orders $w^*_H = \min_{\prec} \{ w_{H,\prec} \}$.
    \end{definition}
    \pause
    \begin{remark}[Complexity of Algorithm~\ref{alg:sum-product-variable-elimination}]
        The complexity of variable elimination is determined by the induced width as $\mathcal{O}(\abs{\mathcal{X}} \cdot (w_{H,\prec} + 1)^L)$, where $L$ is the maximal number of states of variables in $\mathcal{X}$.
    \end{remark}
    \pause
    \begin{remark}[Finding Best Elimination Order]
        Finding the best elimination order will allow to run Algorithm~\ref{alg:sum-product-variable-elimination} with minimal complexity.
        However, finding the best elimination order is NP-hard.
    \end{remark}
\end{frame}

 \begin{frame}{Naive Bayes Model: Tree Width}
     \begin{example}[Naive Bayes Model]
         \begin{minipage}[t]{0.32\textwidth}
 \begin{center}
 Original Bayesian network
 \\ \  \\
 \begin{tikzpicture}[scale=0.8]
 \node[rand_var] (A) at (0,0) {A};
 \foreach \a in {1,2,...,5}{
     \node[rand_var] (\a) at (-3 + \a, -2) {\a};
     \draw[->] (A) to (\a);
 }
 \end{tikzpicture}
 \end{center}
 \end{minipage}
 \pause
 \begin{minipage}[t]{0.32\textwidth}
     \begin{center}
     Elimination order $A \prec 1 \ldots \prec 5$
 \begin{tikzpicture}[scale=0.8]
\node[rand_var] (A) at (0,0) {A};
\node[rand_var] (1) at (-3 + 1, -2) {1};
\node[rand_var] (2) at (-3 + 2, -2) {2};
\node[rand_var] (3) at (-3 + 3, -2) {3};
\node[rand_var] (4) at (-3 + 4, -2) {4};
\node[rand_var] (5) at (-3 + 5, -2) {5};


\draw (1) to (A);
\draw (2) to (A);
\draw (3) to (A);
\draw (4) to (A);
\draw (5) to (A);

\pause
 \draw[red] (1) to (2);
 \draw[red, bend left=40] (1) to (3);
 \draw[red, bend left=45] (1) to (4);
 \draw[red, bend left=50] (1) to node (15) {} (5);
 \draw[red] (2) to (3);
 \draw[red, bend left=40] (2) to (4);
 \draw[red, bend left=45] (2) to node (25) {} (5);
 \draw[red] (3) to (4);
 \draw[red, bend left=40] (3) to node (35) {} (5);
 \draw[red] (4) to (5);
 \filldraw[fill=blue!70, nearly transparent, tension=0.7] plot[smooth cycle] coordinates {(A.north) (1.west) (2.south) (3.south) (4.south) (5.east)};
 \pause 
 \filldraw[fill=green!70, nearly transparent, tension=0.7] plot[smooth cycle] coordinates {(1.west) (1.south) (2.south) (3.south) (4.south) (5.south) (5.east) (5.east) (15.north)};
 \pause 
 \filldraw[fill=orange!70, nearly transparent, tension=0.7] plot[smooth cycle] coordinates {(2.west) (2.south) (3.south) (4.south) (5.south) (5.east) (25.north)};
 \pause 
 \filldraw[fill=red!70, nearly transparent, tension=0.7] plot[smooth cycle] coordinates {(3.west) (3.south) (4.south) (5.south) (5.east) (35.north)};
 \pause 
 \filldraw[fill=brown!70, nearly transparent, tension=0.7] plot[smooth cycle] coordinates {(4.west) (4.south) (5.south) (5.east) (5.north) (4.north)};
 \end{tikzpicture}
 \pause
 $w_{H,\prec} = \pause 5$
 \end{center}
 \end{minipage}
 %
 \begin{minipage}[t]{0.32\textwidth}
     \begin{center}
     Elimination order $1 \ldots \prec 5 \prec A$
 \begin{tikzpicture}[scale=0.8]
\node[rand_var] (A) at (0,0) {A};

\node[rand_var] (1) at (-3 + 1, -2) {1};
\node[rand_var] (2) at (-3 + 2, -2) {2};
\node[rand_var] (3) at (-3 + 3, -2) {3};
\node[rand_var] (4) at (-3 + 4, -2) {4};
\node[rand_var] (5) at (-3 + 5, -2) {5};

\draw (1) to (A);
\draw (2) to (A);
\draw (3) to (A);
\draw (4) to (A);
\draw (5) to (A);

 \pause 
 \filldraw[fill=blue!70, nearly transparent, tension=0.7] plot[smooth cycle] coordinates {(A.north) (A.east) (1.east) (1.south) (1.west) (A.west)};
 \pause 
 \filldraw[fill=green!70, nearly transparent, tension=0.7] plot[smooth cycle] coordinates {(A.north) (A.east) (2.east) (2.south) (2.west) (A.west)};
 \pause 
 \filldraw[fill=orange!70, nearly transparent, tension=0.7] plot[smooth cycle] coordinates {(A.north) (A.east) (3.east) (3.south) (3.west) (A.west)};
 \pause 
 \filldraw[fill=red!70, nearly transparent, tension=0.7] plot[smooth cycle] coordinates {(A.north) (A.east) (4.east) (4.south) (4.west) (A.west)};
 \pause 
 \filldraw[fill=brown!70, nearly transparent, tension=0.7] plot[smooth cycle] coordinates {(A.north) (A.east) (5.east) (5.south) (5.west) (A.west)};
 \end{tikzpicture}
 \pause
 $w_{H,\prec} = \pause 1$
 \end{center}
 \end{minipage}
 \end{example}
 \end{frame}